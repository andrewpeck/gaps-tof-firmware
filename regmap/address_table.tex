\documentclass[9pt,letterpaper]{article}
\usepackage[left=1.5cm, right=1.5cm, top=2cm]{geometry}
\usepackage{ltablex}
\usepackage{makecell}
\usepackage{tabularx}
\renewcommand\familydefault{\sfdefault}
\usepackage[T1]{fontenc}
\usepackage[usenames, dvipsnames]{color}
\definecolor{parentcolor}{rgb}{0.325, 0.408, 0.584}
\definecolor{modulecolor}{rgb}{1.000, 1.000, 1.000}

\date{}

\renewcommand{\contentsname}{Modules}

\usepackage{hyperref}
\setcounter{tocdepth}{4}
\hypersetup{
    colorlinks=true, %set true if you want colored links
    linktoc=all,     %set to all if you want both sections and subsections linked
    linkcolor=black, %choose some color if you want links to stand out
}

\title{UCLA DRS DAQ Address Table}
% START: ADDRESS_TABLE_VERSION :: DO NOT EDIT
    \author{\\ v03.02.05.0C \\ 20190611}
% END: ADDRESS_TABLE_VERSION :: DO NOT EDIT
\begin{document}

\maketitle
%tableofcontents

% START: ADDRESS_TABLE :: DO NOT EDIT

    \pagebreak
    \section{Module: DRS \hfill \texttt{0x0}}

    Implements various control and monitoring functions of the DRS Logic\\

    \renewcommand{\arraystretch}{1.3}
    \noindent
    \subsection*{\textcolor{parentcolor}{\textbf{DRS.CHIP}}}

    \vspace{4mm}
    \noindent
    Registers for configuring the DRS ASIC Directly
    \noindent

    \keepXColumns
    \begin{tabularx}{\linewidth}{ | l | l | r | c | l | X | }
    \hline
    \textbf{Node} & \textbf{Adr} & \textbf{Bits} & \textbf{Perm} & \textbf{Def} & \textbf{Description} \\\hline
    \nopagebreak
    DMODE & \texttt{0x0} & \texttt{[1:1]} & rw & \texttt{0x1} & set 1 = continuous domino, 0=single shot \\\hline
    STANDBY\_MODE & \texttt{0x0} & \texttt{[2:2]} & rw & \texttt{0x0} & set 1 = shutdown drs \\\hline
    TRANSPARENT\_MODE & \texttt{0x0} & \texttt{[3:3]} & rw & \texttt{0x0} & set 1 = transparent mode \\\hline
    DRS\_PLL\_LOCK & \texttt{0x0} & \texttt{[4:4]} & r & \texttt{} & DRS PLL Locked \\\hline
    CHANNEL\_CONFIG & \texttt{0x0} & \texttt{[31:24]} & rw & \texttt{0xFF} & Write Shift Register Configuration                             \\\\ \# of chn - \# of cells per ch - bit pattern                             \\\\ 8        - 1024              - 11111111b                             \\\\ 4        - 2048              - 01010101b                             \\\\ 2        - 4096              - 00010001b                             \\\\ 1        - 8192              - 00000001b \\\hline
    DTAP\_FREQ & \texttt{0x1} & \texttt{[31:0]} & r & \texttt{} & Frequency of DTAP in MHz \\\hline
    \end{tabularx}
    \vspace{5mm}


    \noindent
    \subsection*{\textcolor{parentcolor}{\textbf{DRS.READOUT}}}

    \vspace{4mm}
    \noindent
    Registers for configuring the readout state machine
    \noindent

    \keepXColumns
    \begin{tabularx}{\linewidth}{ | l | l | r | c | l | X | }
    \hline
    \textbf{Node} & \textbf{Adr} & \textbf{Bits} & \textbf{Perm} & \textbf{Def} & \textbf{Description} \\\hline
    \nopagebreak
    ROI\_MODE & \texttt{0x10} & \texttt{[0:0]} & rw & \texttt{0x1} & Set to 1 to enable Region of Interest Readout \\\hline
    BUSY & \texttt{0x10} & \texttt{[1:1]} & r & \texttt{} & Readout is busy \\\hline
    ADC\_LATENCY & \texttt{0x10} & \texttt{[9:4]} & rw & \texttt{0x0} & Latency from first sr clock to when ADC data should be valid \\\hline
    SAMPLE\_COUNT & \texttt{0x10} & \texttt{[21:12]} & rw & \texttt{0x3FF} & Number of samples to read out (0 to 1023) \\\hline
    EN\_SPIKE\_REMOVAL & \texttt{0x10} & \texttt{[22:22]} & rw & \texttt{0x1} & set 1 to enable spike removal \\\hline
    READOUT\_MASK & \texttt{0x11} & \texttt{[7:0]} & rw & \texttt{0xFF} & 8 bit mask, set a bit to 1 to enable readout of that channel. 9th is auto-read if any channel is enabled \\\hline
    START & \texttt{0x12} & \texttt{[0:0]} & w & Pulse & Write 1 to take the state machine out of idle mode \\\hline
    REINIT & \texttt{0x13} & \texttt{[0:0]} & w & Pulse & Write 1 to reinitialize DRS state machine (restores to idle state) \\\hline
    CONFIGURE & \texttt{0x14} & \texttt{[0:0]} & w & Pulse & Write 1 to configure the DRS. Should be done before data taking \\\hline
    DRS\_RESET & \texttt{0x15} & \texttt{[0:0]} & w & Pulse & Write 1 to completely reset the DRS state machine logic \\\hline
    DAQ\_RESET & \texttt{0x16} & \texttt{[0:0]} & w & Pulse & Write 1 to completely reset the DAQ state machine logic \\\hline
    DMA\_RESET & \texttt{0x17} & \texttt{[0:0]} & w & Pulse & Write 1 to completely reset the DMA state machine logic \\\hline
    \end{tabularx}
    \vspace{5mm}


    \noindent
    \subsection*{\textcolor{parentcolor}{\textbf{DRS.FPGA.DNA}}}

    \vspace{4mm}
    \noindent
    FPGA Device DNA
    \noindent

    \keepXColumns
    \begin{tabularx}{\linewidth}{ | l | l | r | c | l | X | }
    \hline
    \textbf{Node} & \textbf{Adr} & \textbf{Bits} & \textbf{Perm} & \textbf{Def} & \textbf{Description} \\\hline
    \nopagebreak
    DNA\_LSBS & \texttt{0x20} & \texttt{[31:0]} & r & \texttt{} & Device DNA [31:0] \\\hline
    DNA\_MSBS & \texttt{0x21} & \texttt{[24:0]} & r & \texttt{} & Device DNA [56:32] \\\hline
    \end{tabularx}
    \vspace{5mm}


    \noindent
    \subsection*{\textcolor{parentcolor}{\textbf{DRS.FPGA.TIMESTAMP}}}

    \vspace{4mm}
    \noindent
    Timestamp
    \noindent

    \keepXColumns
    \begin{tabularx}{\linewidth}{ | l | l | r | c | l | X | }
    \hline
    \textbf{Node} & \textbf{Adr} & \textbf{Bits} & \textbf{Perm} & \textbf{Def} & \textbf{Description} \\\hline
    \nopagebreak
    TIMESTAMP\_LSBS & \texttt{0x24} & \texttt{[31:0]} & r & \texttt{} & Device TIMESTAMP [31:0] \\\hline
    TIMESTAMP\_MSBS & \texttt{0x25} & \texttt{[15:0]} & r & \texttt{} & Device TIMESTAMP [47:32] \\\hline
    \end{tabularx}
    \vspace{5mm}


    \noindent
    \subsection*{\textcolor{parentcolor}{\textbf{DRS.DAQ}}}

    \vspace{4mm}
    \noindent
    DAQ
    \noindent

    \keepXColumns
    \begin{tabularx}{\linewidth}{ | l | l | r | c | l | X | }
    \hline
    \textbf{Node} & \textbf{Adr} & \textbf{Bits} & \textbf{Perm} & \textbf{Def} & \textbf{Description} \\\hline
    \nopagebreak
    INJECT\_DEBUG\_PACKET & \texttt{0x30} & \texttt{[0:0]} & w & Pulse & Injects a fixed format debug packet into the DAQ \\\hline
    \end{tabularx}
    \vspace{5mm}


    \noindent
    \subsection*{\textcolor{parentcolor}{\textbf{DRS.Trigger}}}

    \vspace{4mm}
    \noindent
    Trigger
    \noindent

    \keepXColumns
    \begin{tabularx}{\linewidth}{ | l | l | r | c | l | X | }
    \hline
    \textbf{Node} & \textbf{Adr} & \textbf{Bits} & \textbf{Perm} & \textbf{Def} & \textbf{Description} \\\hline
    \nopagebreak
    FORCE\_TRIGGER & \texttt{0x40} & \texttt{[0:0]} & w & Pulse & Generates a trigger \\\hline
    EN\_EXT\_TRIGGER & \texttt{0x41} & \texttt{[0:0]} & rw & \texttt{0x1} & Generates a trigger \\\hline
    \end{tabularx}
    \vspace{5mm}


    \noindent
    \subsection*{\textcolor{parentcolor}{\textbf{DRS.COUNTERS}}}

    \vspace{4mm}
    \noindent
    Counters
    \noindent

    \keepXColumns
    \begin{tabularx}{\linewidth}{ | l | l | r | c | l | X | }
    \hline
    \textbf{Node} & \textbf{Adr} & \textbf{Bits} & \textbf{Perm} & \textbf{Def} & \textbf{Description} \\\hline
    \nopagebreak
    CNT\_SEM\_CORRECTION & \texttt{0x50} & \texttt{[15:0]} & r & \texttt{} & Number of Single Event Errors corrected by the scrubber \\\hline
    CNT\_SEM\_UNCORRECTABLE & \texttt{0x51} & \texttt{[19:16]} & r & \texttt{} & Number of Critical Single Event Errors (uncorrectable by scrubber) \\\hline
    CNT\_READOUTS\_COMPLETED & \texttt{0x52} & \texttt{[31:0]} & r & \texttt{} & Number of readouts completed since reset \\\hline
    CNT\_DMA\_READOUTS\_COMPLETED & \texttt{0x53} & \texttt{[31:0]} & r & \texttt{} & Number of readouts completed since reset \\\hline
    CNT\_LOST\_EVENT & \texttt{0x54} & \texttt{[31:16]} & r & \texttt{} & Number of trigger lost due to deadtime \\\hline
    CNT\_EVENT & \texttt{0x55} & \texttt{[31:0]} & r & \texttt{} & Number of triggers received \\\hline
    \end{tabularx}
    \vspace{5mm}


    \noindent
    \subsection*{\textcolor{parentcolor}{\textbf{DRS.HOG}}}

    \vspace{4mm}
    \noindent
    HOG Parameters
    \noindent

    \keepXColumns
    \begin{tabularx}{\linewidth}{ | l | l | r | c | l | X | }
    \hline
    \textbf{Node} & \textbf{Adr} & \textbf{Bits} & \textbf{Perm} & \textbf{Def} & \textbf{Description} \\\hline
    \nopagebreak
    GLOBAL\_DATE & \texttt{0x60} & \texttt{[31:0]} & r & \texttt{} & HOG Global Date \\\hline
    GLOBAL\_TIME & \texttt{0x61} & \texttt{[31:0]} & r & \texttt{} & HOG Global Time \\\hline
    GLOBAL\_VER & \texttt{0x62} & \texttt{[31:0]} & r & \texttt{} & HOG Global Version \\\hline
    GLOBAL\_SHA & \texttt{0x63} & \texttt{[31:0]} & r & \texttt{} & HOG Global SHA \\\hline
    TOP\_SHA & \texttt{0x64} & \texttt{[31:0]} & r & \texttt{} & HOG Top SHA \\\hline
    TOP\_VER & \texttt{0x65} & \texttt{[31:0]} & r & \texttt{} & HOG Top Version \\\hline
    HOG\_SHA & \texttt{0x66} & \texttt{[31:0]} & r & \texttt{} & HOG SHA \\\hline
    HOG\_VER & \texttt{0x67} & \texttt{[31:0]} & r & \texttt{} & HOG Version \\\hline
    \end{tabularx}
    \vspace{5mm}


    \noindent
    \subsection*{\textcolor{parentcolor}{\textbf{DRS.SPY}}}

    \vspace{4mm}
    \noindent
    Spy Buffer
    \noindent

    \keepXColumns
    \begin{tabularx}{\linewidth}{ | l | l | r | c | l | X | }
    \hline
    \textbf{Node} & \textbf{Adr} & \textbf{Bits} & \textbf{Perm} & \textbf{Def} & \textbf{Description} \\\hline
    \nopagebreak
    RESET & \texttt{0x70} & \texttt{[0:0]} & w & Pulse & Spy Buffer Reset \\\hline
    DATA & \texttt{0x71} & \texttt{[15:0]} & r & \texttt{} & Spy Read Data \\\hline
    FULL & \texttt{0x72} & \texttt{[0:0]} & r & \texttt{} & Spy Buffer Full \\\hline
    EMPTY & \texttt{0x72} & \texttt{[1:1]} & r & \texttt{} & Spy Buffer Empty \\\hline
    \end{tabularx}
    \vspace{5mm}


% END: ADDRESS_TABLE :: DO NOT EDIT

\end{document}


